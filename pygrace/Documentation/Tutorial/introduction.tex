PyGrace is a Python package used for creating Grace files. Grace is a
simple 2-D plotting program.  PyGrace is intended to:
\begin{itemize}
\item enable Grace ({\tt .agr} format) file creation within Python.
\item assist in the automated creation of many figures for use in
  large-scale, exploratory data analysis.
\item encourage code reuse, by offering an object-oriented structure.
\item reduce the time required to create figures for scientific
  journal articles.
\end{itemize}
Visualizing data is often an integral part of the data analysis
process, however, \textbf{PyGrace is explicity intended not to perform
  any type of data analysis}.  Not even computing an average.
Especially not performing a linear regression.  This principle
underlies much of the structure of PyGrace, with the purpose of not
allowing users to adopt a ``black box'' approach to analysis.  While
visualizing data should be as painless as possible, the pain required
to understand the analysis process is considered necessary, or even
desirable (if you're into that sort of thing).

This tutorial is meant to help with installation of PyGrace, and give
new users an idea of how PyGrace is structured.  After reading this
brief document, the next step for learning to use PyGrace is to go
through the examples that are located in the {\tt PyGrace/Examples}
directory.
